%Copyright (c) 2004 2005 2006 Atos Origin
%Permission is granted to copy, distribute and/or modify this
%document
%under the terms of the GNU Free Documentation License,
%      Version 1.2
%      or any later version published by the Free Software
%      Foundation;
%      with no Invariant Sections, no Front-Cover
%      Texts, and no Back-Cover
%      Texts.  A copy of the license is
%      included in the section entitled "GNU
%      Free Documentation License".
%
%$Id: glossary.tex,v 1.1 2006/02/16 16:33:33 goneri Exp $
\section*{Glossary}
%\addcontentsline{toc}{chapter}{Glossaire}


\paragraph{Fork}
A fork is an event which sometimes occurs in a development project, typically in 
Community projects (as in many free and open source software), when opinions 
within the development team diverge on the direction to be taken and prove to be incompatible. 
The development of the software splits then in two different directions, 
at the instigation of both parties.

\paragraph{O3S}
Open Source Selection Software. Tool developed by Atos Origin, implementing the 
QSOS method, which will be used on the \url{http://www.qsos.org} website to create, 
modify and consult ID cards and evaluation sheets.

\paragraph{QSOS method}
Method for Qualification and Selection of Open Source software, designed and used by Atos Origin as a basis of its support and technological watch services. It is made available - under the terms of a free licence - on the  \url{http://www.qsos.org} website.

\paragraph{Service provider}
Any company desiring to offer services around free and open source software (expertise, integration, development, support...).

\paragraph{User}
Any person, entity, company or administration using or planning to use free or open source software.




